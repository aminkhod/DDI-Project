% !TeX root=main.tex
% در این فایل، عنوان پایان‌نامه، مشخصات خود، متن تقدیمی‌، ستایش، سپاس‌گزاری و چکیده پایان‌نامه را به فارسی، وارد کنید.
% توجه داشته باشید که جدول حاوی مشخصات پایان‌نامه/رساله و همچنین، مشخصات داخل آن، به طور خودکار، درج می‌شود.
%%%%%%%%%%%%%%%%%%%%%%%%%%%%%%%%%%%%
% دانشگاه خود را وارد کنید
\university{شهید بهشتی}
% دانشکده، آموزشکده و یا پژوهشکده  خود را وارد کنید
\faculty{علوم ریاضی}
% مقطع تحصیلی خود (کارشناسی ارشد یا دکتری) را وارد کنید
\degree {کارشناسی ارشد} 
% گروه آموزشی خود را وارد کنید
\subject{علوم کامپیوتر }
% گرایش خود را وارد کنید
\field{محاسبات علمی}
% عنوان پایان‌نامه را وارد کنید
\title{
پیش‌بینی برهم‌کنش جفت داروها با رویکرد سیستم‌های توصيه‌گر
}
% نام استاد(ان) راهنما را وارد کنید
\firstsupervisor{دکتر چنگیز اصلاحچی}
%\secondsupervisor{استاد راهنمای دوم}
% نام استاد(دان) مشاور را وارد کنید. چنانچه استاد مشاور ندارید، دستور پایین را غیرفعال کنید.
%\firstadvisor{}
%\secondadvisor{}
% نام پژوهشگر را وارد کنید
\name{بهاره}
% نام خانوادگی پژوهشگر را وارد کنید
\surname{لويان}
% تاریخ پایان‌نامه را وارد کنید
\thesisdate{۱۳۹۹}
% کلمات کلیدی پایان‌نامه را وارد کنید
\keywords{برهم‌کنش دارو-دارو، شباهت دارویی، ادغام شباهت‌های دارويی، انتخاب ویژگی، اعتبارسنجی متقابل، سیستم توصيه‌گر
}
% چکیده پایان‌نامه را وارد کنید
\fa-abstract{\noindent	
برهم‌کنش دارو-دارو ممکن است باعث بروز واکنش‌های دارویی نامطلوب یا حتی تاثیر دارویی منفی شود. لذا شناسایی برهم‌کنش دارو-دارو قبل از تجویز چند دارو، بسیار مهم است. شناسایی بالینی برهم‌کنش دارو-دارو به هزینه و زمان زیادی نیاز دارد، رویکردهای محاسباتی غربالگری به‌عنوان گزینه جایگزین، روش بسیار ارزان‌تری را برای شناسایی برهم‌کنش‌های بالقوه در مقیاس بزرگ فراهم می کنند. با این وجود اکثر آنها فقط پیش بینی می‌کنند که آیا یک دارو بر داروی دیگر برهم‌کنش دارد یا خیر، اما از تاثیرات افزاینده (مثبت) و کاهنده (منفی) آنها غافل می‌شوند. این‌گونه برهم‌کنش‌های جامع سه کلاسه به‌طور تصادفی اتفاق نمی‌افتند و برآمده از ویژگی‌های ساختاری گراف برهم‌کنش دارو-دارو است.از طرفی آشکار ساختن چنین رابطه‌هایی بسیار مهم است زیرا به درک و فهم نحوه عملکرد برهم‌کنش دارو-داروهای مرتبه بالا کمک می‌کند و راهنمای مهمی در تجویز نسخه واحد می‌باشد.
\newline
در این کار با استفاده از روابط ساختاری ذاتی بین داروها، مجموعه‌ای از برهم‌کنش دارو-داروی جامع را به عنوان یک ماتریس مشخص در نظر گرفته‌ایم و سپس برای پیش‌بینی‌های افزاینده و کاهنده براساس سیستم‌های توصیه‌گر یک مدل جدید ترکیب ویژگی‌ها و یادگیری عمیق به نام
\lr{SNF-CNN}
طراحی کرده‌ایم. از ارزیابی نتایج
\lr{SNF-CNN}  
در پیش‌بینی برهم‌کنش دارو-دارو مقادیر
$AUC=0/9747 \pm 0/0033 $
و
$AUPR=0/9666 \pm 0/0045 $
برای جفت داروهای کاهنده، مقادیر
$AUC=0/9686 \pm 0/0028$
و
$AUPR=0/8221 \pm 0/0184$
برای جفت داروهای افزاینده و مقادیر
$AUC=0/9714 \pm 0/0040$
و
$AUPR=0/9480 \pm 0/0083$
برای عدم برهم‌کنش‌ها بدست آمد. که در مقایسه با مقالات برتر در این زمینه، برتری خود را نشان داد. شایان ذکر است این رویکرد جدید نه تنها قادر به پیش‌بینی برهم‌کنش دارو - دارو جامع سه کلاسه است بلکه برهم‌کنش دارو-دارو دوکلاسه معمولی را نیز پیش‌بینی می‌کند.  
\newline
\newline
\newline
منابع اصلی:
\newline
	\begin{latin}
	\begin{enumerate}
		\item Shi J-Y, Huang H, Li J-X, Lei P, Zhang Y-N, Dong K, Yiu S-M (2018) TMFUF: a triple matrix factorization-based unified framework for predicting comprehensive drug–drug interactions of new drugs. BMC Bioinform 19(S14):411.
		\item Yu H, Mao K-T, Shi J-Y, Huang H, Chen Z, Dong K, Yiu S-M.Predicting and understanding comprehensive drug–drug interactions via seminonnegative matrix factorization.BMC Syst Biol2018;12:14.	
		\item Shi J. Mao K.Yu H. et al.Detecting drug communities and predicting comprehensive drug–drug interactions via balance regularized semi-nonnegative matrix factorization.J Cheminform2019;11: 28.
	\end{enumerate}
	\end{latin}
}


\newpage
\thispagestyle{empty}
\vtitle
\newpage

\thispagestyle{empty}
\clearpage
~~~
\newpage
\thispagestyle{empty}
% !TeX root=main.tex
\ \\ \\ \\ \\ \\ \\ \\
{\XBSols
	\addfontfeature{Scale=1.5}

\vspace*{5 cm}

\centerline{كلية حقوق اعم از چاپ و تكثير، نسخه برداري ، ترجمه، اقتباس و ... از }
\centerline{اين پايان نامه براي دانشگاه شهيد بهشتي محفوظ است.}
\vspace*{1 cm}
\centerline{نقل  مطالب با ذكر مأخذ آزاد است.}
}
\newpage
\thispagestyle{empty}
\clearpage
~~~
%\newpage
%\thispagestyle{empty}
%\centerline{{\includegraphics[width=20 cm]{replyrecord}}}
%\newpage
%\thispagestyle{empty}
%\clearpage
%~~~
\newpage
 % پایان‌نامه خود را تقدیم کنید!
\begin{acknowledgementpage}

\vspace{4cm}

{\nastaliq
{\Large
 تقدیم به تمام عزیزانم،
\vspace{1.5cm}

\newdimen\xa
\xa=\textwidth
\advance \xa by -11cm
\hspace{\xa}
به پاس یک عمر اخلاص، فداکاری و محبتشان...
}}
\end{acknowledgementpage}
\newpage
\thispagestyle{empty}
\clearpage
~~~
%%%%%%%%%%%%%%%%%%%%%%%%%%%%%%%%%%%%

%\newpage
%\thispagestyle{empty}
%\clearpage
%~~~
%%%%%%%%%%%%%%%%%%%%%%%%%%%%%%%%%%%%
\newpage
\thispagestyle{empty}
% سپاس‌گزاری
{\nastaliq
سپاس‌گزاری...
}
\\[2cm]

شکر و سپاس خدا را که بزرگترین امید و یاور در لحظه لحظه زندگیست.

با تقدیر و تشکر از زحمات استاد فرهیخته و فرزانه آقای دکتر چنگیز اصلاحچی که همواره راهنما و راه‌گشای من در به نتیجه رسیدن این پایان‌نامه بوده اند.


% با استفاده از دستور زیر، امضای شما، به طور خودکار، درج می‌شود
\signature
%{\small
\abstractview
%}
\newpage
\thispagestyle{empty}
\clearpage
~~~
