% !TeX root=main.tex
\addcontentsline{toc}{chapter}{پیشگفتار}
\pagenumbering{harfi}
\chapter*{پیش‌گفتار}
\markboth{پیش‌گفتار}{پیش‌گفتار}

 برهم‌کنش دارو-دارو یکی از مهمترین موضوعات در زمینه توسعه دارو و سلامت است. برهم‌کنش دارو-دارو به‌عنوان عوارض‌جانبی ناخواسته ناشی از مصرف هم‌زمان دو یا چند داروی شناخته می‌باشد که می‌تواند تاثیر مصرف دارو را کاهش دهد و یا مسمومیت غیرمنتظره ایجاد کند. در مواردی كه پزشک چندین دارو را به‌طور هم‌زمان برای بیمار تجویز کند، ممكن است برهم‌کنش دارو-دارو عوارض‌جانبی جبران ناپذیری ایجاد کند. تاثیر داروها بر روی یکدیگر ممکن است به بیماری‌های دیگر یا حتی مرگ منجر شود. این عوارض‌جانبی به‌ویژه در افراد بزرگسال و بیماران سرطانی که روزانه مقدار زیادی دارو مصرف می‌کنند، بسیار قابل توجه است.

تعداد زیادی از تحقیقات حال حاضر در مطالعات بالینی بر روی برهم‌کنش دارو-دارو متمرکز هستند. رویکردهای تجربی و سنتی برای آزمودن و پیدا کردن برهم‌کنش دارو-دارو از روش‌های آزمایشگاهی و بالینی استفاده می‌کنند، که این روند با چالش‌های بسیاری روبروست: هزینه‌ی مالی و زمانی زیادی را می‌طلبد، رفاه حال حیوانات نادیده گرفته ‌می‌شود و تعداد متقاضیان شرکت در طرح‌های آزمایشی کم هستند. برای کمک به کاهش هزینه‌ها و تجزیه و تحلیل تعداد بیشتری از برهم‌کنش‌ها دارو-دارو به روشی جامع و خودکار برای پیش‌بینی نیاز است. رویکردهای محاسباتی، جایگزین امیدوارکنند‌ه‌ای برای کشف و تشخیص برهم‌کنش دارو-دارو در مقیاس وسیع هستند که به‌تازگی از طرف مراکز تحقیقاتی و صنعت مورد توجه قرار گرفته‌اند.

یکی از مسائل مهم برای ارائه روش پیش‌بینی برهم‌کنش‌ دارو-دارو، استفاده از داده‌هایی مناسب است که بتواند مقایسه‌ای عادلانه با سایر روش‌ها داشته باشد.  در این پایان‌نامه ما از مجموعه داده‌ای شامل دو نوع برهم‌کنش کاهنده و افزاینده برای بررسی و مقایسه‌ی آن با سایر روش‌ها استفاده کرده‌ایم.

سیستم‌های توصیه‌گر با اولین ظهورشان در زمینه‌ی پالایش گروهی، حوزه تحقیقاتی مهمی را در اواسط دهه ١٩٩٠ فراهم نمودند. در این پایان نامه  الگوریتمی برای پیش‌بینی برهم‌کنش دارو-دارو بر مبنای سیستم‌های توصیه‌گر ارائه شده است.
 