\chapter{ کار‌های آینده و مسائل باز}
\label{chapter:conclusion}

رویکردهای محاسباتی موجود قادر به ارائه‌ی برهم‌کنش‌های بالقوه  در مقیاس بزرگ قبل از استفاده از داروها در بازار هستند. با این حال نمی‌توانند برهم‌کنش‌های جامع، شامل برهم‌کنش‌های افزینده و کاهنده را پیش‌بینی کنند. دانستن نوع برهم‌کنش افزاینده یا کاهنده یک جفت دارو ، از این که فقط بدانیم یک جفت دارو برهم‌کنش دارند، مفیدتر است. اکثر رویکردهای موجود بدون درنظر گرفتن تغییرات دارویی ناشی از برهم‌کنش، فقط پیش‌بینی دودویی را گزارش می‌دهند. علاوه‌بر این، بروز برهم‌کنش‌های افزاینده و کاهنده تصادفی نیست، اما هیچ یک از رویکردهای موجود از این خاصیت ذاتی مهم برهم‌کنش‌ها در هنگام درمان بیماری‌های پیچیده (شامل درمان با سه یا چند دارو)، مورد بررسی و استفاده قرار نمی‌دهد.

در این پایان‌نامه پس از ارائه‌ی شبکه‌ی جامع برهم‌کنش‌ها، از ساختار سیستم‌های توصیه‌گر برای طراحی یک الگوریتم جدید استفاده کردیم. اگرچه پیش‌بینی به‌دست آمده توسط الگوریتم جدید الهام بخش است، اما عملکرد کلی هنوز هم می‌تواند بهبود یابد. به‌همین دلیل برهم‌کنش‌های نادرست پیش‌بینی شده را بررسی می کنیم. پس از بررسی آنها به‌صورت موردی، و در جهت اثبات عملی الگوریتم به بررسی عملکرد پیش‌‌بینی الگوریتم در آخرین نسخه‌ی پایگاه داده‌ی 
\lr{Drug Bank}
 پرداختیم. مشاهدات و بررسی‌ها منجر به کشف دو علت برای پیش‌بینی‌های نادرست شد. 

۱) تعدادی جفت داروی مثبت کاذب که در نسخه‌ی چهار 
\lr{Drug Bank}
به‌طور دقیق به‌عنوان برهم‌کنش برچسب‌گذاری شده‌اند اما در نسخه فعلی به‌درستی به‌عنوان غیر برهم‌کنش شناخته می‌شوند. به‌عنوان مثال، در نسخه قدیمی
\lr{Drug Bank}
ثبت شده است که
\lr{Apraclonidine}
(داروی مقلد سمپاتیک
\LTRfootnote{Sympathomimetic}
 مورد استفاده در درمان آب‌سیاه
\LTRfootnote{Glaucoma} 
) در استفاده هم‌زمان با 
\lr{Alprenolol}
و
\lr{Bevantolol}
فعالیت‌های انسداد دهلیزی
\LTRfootnote{Atrioventricular Blocking}
را افزایش می‌دهد، درحالی‌که نسخه جدید آن را حذف می کند.

 
2) جفت داروهای منفی کاذب که در نسخه‌ی چهار
\lr{Drug Bank}
به‌ اشتباه به‌عنوان غیربرهم‌کنش برچسب‌گذاری شده‌اند اما در نسخه فعلی به‌عنوان برهم‌کنش شناخته می‌شوند.

به‌عنوان مثال، جفت داروی
\lr{Valrubicin}
و
\lr{Cyclosporine}
و همچنین جفت داروی
\lr{Ergocalciferol}
و
\lr{Calcitriol}.
 نسخه جدید
\lr{Drug Bank} 
گزارش می‌کند، 
\lr{Valrubicin}
(داروی درمان سرطان مثانه) باعث افزایش فعالیت داروی نفروتوکسیک
\LTRfootnote{Nephrotoxic} 
\lr{Cyclosporine}
(داروی سرکوب‌کننده سیستم ایمنی بدن با عمل خاص روی 
\lr{T-lymphocytes}
) می‌شود. درمان ترکیبی
\lr{Calcitriol}
و
\lr{Ergocalciferol}
ریسک یا شدت عوارض‌جانبی را در درمان چند دارویی افزایش می‌دهد. 

همچنین مشاهدات بیشتر نشان داد بعضی برهم‌کنش‌های موجود در نسخه‌ی چهار در نسخه پنج دیگر برهم‌کنش نیستند. مشاهدات ما را مقاله‌ی
\cite{Shi J2019}
تایید می‌کند.
پیش‌بینی می‌شود در آینده، مدل‌های پیش‌بینی موجود با جمع‌آوری مجموعه داده‌های بهتر و بیش‌تر بهبود یابند. مجموعه داده‌های جدیدتر تعداد کمتری از هر دو نوع جفت داروهای مثبت و منفی کاذب جمع‌آوری خواهند‌ کرد. برای کارهای بعدی پیشنهاد می‌شود مجموعه داده همواره از آخرین نسخه‌ی
\lr{Drug Bank}
جمع‌آوری شود. 

داده‌ی سه‌تایی تلاشی برای بهبود در بیان و حل مسئله نسبت به حالت دودویی است. با این وجود داده‌ها‌ی سه‌تایی نیز به اندازه‌ی کافی معناداری زیستی ندارد و اطلاعات زیستی محدودی ارائه می‌دهد. بدین معنی که پیش‌بینی نوع برهم‌کنش می‌تواند مفید باشد اما مشخص نمی‌شود که این برهم‌کنش در چه مرحله‌ای از مراحل فارماکوکنتیک یا فارماکودینامیک رخ داده است. لذا پیشنهاد می‌شود داد‌ه‌هایی با برچسب کاهنده و افزاینده از هریک از مراحل فارماکوکنتیک و فارماکودینامیک جمع‌آوری شود. در این حالت مدل‌های معنی‌دارتری از نظر داروشناسی و هم یادگیری ماشین آموزش داده می‌شوند. مدل‌های حاصل برای داروشناس و داروساز اهمیت بالاتری داشته و قابل استفاده‌تر خواهد بود.
