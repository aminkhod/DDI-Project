%% BioMed_Central_Tex_Template_v1.06
%%                                      %
%  bmc_article.tex            ver: 1.06 %
%                                       %

%%IMPORTANT: do not delete the first line of this template
%%It must be present to enable the BMC Submission system to
%%recognise this template!!

%%%%%%%%%%%%%%%%%%%%%%%%%%%%%%%%%%%%%%%%%
%%                                     %%
%%  LaTeX template for BioMed Central  %%
%%     journal article submissions     %%
%%                                     %%
%%          <8 June 2020>              %%
%%                                     %%
%%                                     %%
%%%%%%%%%%%%%%%%%%%%%%%%%%%%%%%%%%%%%%%%%

%%%%%%%%%%%%%%%%%%%%%%%%%%%%%%%%%%%%%%%%%%%%%%%%%%%%%%%%%%%%%%%%%%%%%
%%                                                                 %%
%% For instructions on how to fill out this Tex template           %%
%% document please refer to Readme.html and the instructions for   %%
%% authors page on the biomed central website                      %%
%% https://www.biomedcentral.com/getpublished                      %%
%%                                                                 %%
%% Please do not use \input{...} to include other tex files.       %%
%% Submit your LaTeX manuscript as one .tex document.              %%
%%                                                                 %%
%% All additional figures and files should be attached             %%
%% separately and not embedded in the \TeX\ document itself.       %%
%%                                                                 %%
%% BioMed Central currently use the MikTex distribution of         %%
%% TeX for Windows) of TeX and LaTeX.  This is available from      %%
%% https://miktex.org/                                             %%
%%                                                                 %%
%%%%%%%%%%%%%%%%%%%%%%%%%%%%%%%%%%%%%%%%%%%%%%%%%%%%%%%%%%%%%%%%%%%%%

%%% additional documentclass options:
%  [doublespacing]
%  [linenumbers]   - put the line numbers on margins

%%% loading packages, author definitions

%\documentclass[twocolumn]{bmcart}% uncomment this for twocolumn layout and comment line below
\documentclass{bmcart}
\usepackage{hyperref}

%%% Load packages
\usepackage{amsthm,amsmath}
%\RequirePackage[numbers]{natbib}
%\RequirePackage[authoryear]{natbib}% uncomment this for author-year bibliography
%\RequirePackage{hyperref}
\usepackage[utf8]{inputenc} %unicode support
%\usepackage[applemac]{inputenc} %applemac support if unicode package fails
%\usepackage[latin1]{inputenc} %UNIX support if unicode package fails

%%%%%%%%%%%%%%%%%%%%%%%%%%%%%%%%%%%%%%%%%%%%%%%%%
%%                                             %%
%%  If you wish to display your graphics for   %%
%%  your own use using includegraphic or       %%
%%  includegraphics, then comment out the      %%
%%  following two lines of code.               %%
%%  NB: These line *must* be included when     %%
%%  submitting to BMC.                         %%
%%  All figure files must be submitted as      %%
%%  separate graphics through the BMC          %%
%%  submission process, not included in the    %%
%%  submitted article.                         %%
%%                                             %%
%%%%%%%%%%%%%%%%%%%%%%%%%%%%%%%%%%%%%%%%%%%%%%%%%

\def\includegraphic{}
\def\includegraphics{}

%%% Put your definitions there:
\startlocaldefs
\endlocaldefs

%%% Begin ...
\begin{document}

%%% Start of article front matter
\begin{frontmatter}

\begin{fmbox}
\dochead{Research}

%%%%%%%%%%%%%%%%%%%%%%%%%%%%%%%%%%%%%%%%%%%%%%
%%                                          %%
%% Enter the title of your article here     %%
%%                                          %%
%%%%%%%%%%%%%%%%%%%%%%%%%%%%%%%%%%%%%%%%%%%%%%

\title{Predicting Comperhensive Drug - Drug Interaction via Similarity Network Fusion and Convolutional Neural Networks}

%%%%%%%%%%%%%%%%%%%%%%%%%%%%%%%%%%%%%%%%%%%%%%
%%                                          %%
%% Enter the authors here                   %%
%%                                          %%
%% Specify information, if available,       %%
%% in the form:                             %%
%%   <key>={<id1>,<id2>}                    %%
%%   <key>=                                 %%
%% Comment or delete the keys which are     %%
%% not used. Repeat \author command as much %%
%% as required.                             %%
%%                                          %%
%%%%%%%%%%%%%%%%%%%%%%%%%%%%%%%%%%%%%%%%%%%%%%
\author[
  addressref={aff1,aff2},
  email={khodamoradi1992@gmail.com}
]{\inits{M.A.K.}\fnm{Mohammad.Amin} \snm{Khodamoradi}}
\author[
  addressref={aff1,aff2},
  email={Bahar.levian@gmail.com}
]{\inits{B.L.}\fnm{Bahareh} \snm{Levian}}
\author[
  addressref={aff1,aff2},                   % id's of addresses, e.g. {aff1,aff2}
  corref={aff1},                       % id of corresponding address, if any
% noteref={n1},                        % id's of article notes, if any
  email={eslahchi.ch@gmail.com}   % email address
]{\inits{C.E.}\fnm{Changiz} \snm{Eslahchi}}


%%%%%%%%%%%%%%%%%%%%%%%%%%%%%%%%%%%%%%%%%%%%%%
%%                                          %%
%% Enter the authors' addresses here        %%
%%                                          %%
%% Repeat \address commands as much as      %%
%% required.                                %%
%%                                          %%
%%%%%%%%%%%%%%%%%%%%%%%%%%%%%%%%%%%%%%%%%%%%%%

\address[id=aff1]{%                           % unique id
  \orgdiv{Department of Computer Science},             % department, if any
  \orgdiv{Faculty of Mathematical Science},
  \orgname{Shahid Beheshti University},          % university, etc
  \city{Tehran},                              % city
  \cny{Iran}                                    % country
}
\address[id=aff2]{%
  \orgdiv{School of Bioinformatics},
  \orgname{IPM - Institute for Research in Fundamental Sciences},
  %\street{},
  %\postcode{}
  \city{Tehran},                              % city
  \cny{Iran}
}

%%%%%%%%%%%%%%%%%%%%%%%%%%%%%%%%%%%%%%%%%%%%%%
%%                                          %%
%% Enter short notes here                   %%
%%                                          %%
%% Short notes will be after addresses      %%
%% on first page.                           %%
%%                                          %%
%%%%%%%%%%%%%%%%%%%%%%%%%%%%%%%%%%%%%%%%%%%%%%

%\begin{artnotes}
%%\note{Sample of title note}     % note to the article
%\note[id=n1]{Equal contributor} % note, connected to author
%\end{artnotes}

\end{fmbox}% comment this for two column layout

%%%%%%%%%%%%%%%%%%%%%%%%%%%%%%%%%%%%%%%%%%%%%%%
%%                                           %%
%% The Abstract begins here                  %%
%%                                           %%
%% Please refer to the Instructions for      %%
%% authors on https://www.biomedcentral.com/ %%
%% and include the section headings          %%
%% accordingly for your article type.        %%
%%                                           %%
%%%%%%%%%%%%%%%%%%%%%%%%%%%%%%%%%%%%%%%%%%%%%%%

\begin{abstractbox}

\begin{abstract} % abstract
\parttitle{Background} %if any
Drug-drug interactions (DDIs) always cause unexpected and even adverse drug reactions. It is important to identify DDIs before drugs are used in the market.However, preclinical identification of DDIs requires much money and time. Computational approaches have exhibited their abilities to predict potential DDIs on a large scale by utilizing premarket drug properties. Nevertheless, most of them only predict whether or not one drug interacts with another, but neglect their enhancive (positive) and depressive (negative) changes of pharmacological effects. Moreover, these comprehensive DDIs do not occur at random, and derived from the structural features of the graph of DDIs. Revealing such a relationship is very important, because it is able to help understand how DDIs occur. Both the prediction of comprehensive DDIs and the discovery of structural relationship among them play an important guidance when making a co-prescription.

\parttitle{Results} %if any
In this work, treating a set of comprehensive DDIs as a signed network, we design a novel model (SNF-CNN) for the prediction of enhancive and degressive DDIs based on similarity network fusion and convolutional neural networks. SNF-CNN achieves the depressive DDI prediction ($AUC=0/9747 \pm 0/0033 $ and $AUPR=0/9666 \pm 0/0045 $), enhancive DDI prediction ($AUC=0/9686 \pm 0/0028$ and $AUPR=0/8221 \pm 0/0184$) and the Unknown DDI prediction ($AUC=0/9714 \pm 0/0040$ and $AUPR=0/9480 \pm 0/0083$). Compared with three state-of-the-art approaches, SNF-CNN shows it superiority.

\parttitle{Conclusions} %if any
This new approach is not only able to predict comprehensive DDI, but also predicts non-DDI.
\end{abstract}

%%%%%%%%%%%%%%%%%%%%%%%%%%%%%%%%%%%%%%%%%%%%%%
%%                                          %%
%% The keywords begin here                  %%
%%                                          %%
%% Put each keyword in separate \kwd{}.     %%
%%                                          %%
%%%%%%%%%%%%%%%%%%%%%%%%%%%%%%%%%%%%%%%%%%%%%%

\begin{keyword} 
\kwd{Drug-Drug Interaction}
\kwd{Drug Similarity}
\kwd{Drug Similarity Integration}
\kwd{Feature Selection}
\kwd{Recommender System}
\end{keyword}

% MSC classifications codes, if any
%\begin{keyword}[class=AMS]
%\kwd[Primary ]{}
%\kwd{}
%\kwd[; secondary ]{}
%\end{keyword}

\end{abstractbox}
%
%\end{fmbox}% uncomment this for two column layout

\end{frontmatter}

%%%%%%%%%%%%%%%%%%%%%%%%%%%%%%%%%%%%%%%%%%%%%%%%
%%                                            %%
%% The Main Body begins here                  %%
%%                                            %%
%% Please refer to the instructions for       %%
%% authors on:                                %%
%% https://www.biomedcentral.com/getpublished %%
%% and include the section headings           %%
%% accordingly for your article type.         %%
%%                                            %%
%% See the Results and Discussion section     %%
%% for details on how to create sub-sections  %%
%%                                            %%
%% use \cite{...} to cite references          %%
%%  \cite{koon} and                           %%
%%  \cite{oreg,khar,zvai,xjon,schn,pond}      %%
%%                                            %%
%%%%%%%%%%%%%%%%%%%%%%%%%%%%%%%%%%%%%%%%%%%%%%%%

%%%%%%%%%%%%%%%%%%%%%%%%% start of article main body
% <put your article body there>

%%%%%%%%%%%%%%%%
%% Background %%
%%
\section*{Introduction}
When two or more drugs are taken together, drugs' effects or behaviors are unexpectedly influenced by each other
\cite{wienkers2005predicting}. 
This kind of influence is termed as Drug-Drug interaction (DDI), which would reduce drug efficacy, increase unexpected toxicity, or induce other adverse drug reactions between the co-prescribed drugs. As the number of approved drugs increases, the number of drug-unidentified DDIs is rapidly increasing, such that among approved small molecular drugs in Drug Bank, on average, 15 out of every 100  drug pairs have DDIs
\cite{law2014drugbank}. 
The DDIs would put patients who are treated with multiple drugs in an unsafe situation
\cite{leape1995systems, businaro2013we, karbownik2017pharmacokinetic, mulroy2017giant}.
Understanding DDI is the first step in drug combinations, which becomes one of the most promising solutions for the treatment of multifactorial complex diseases
\cite{zhao2011prediction}.
Therefore, there is an urgent need for screening and analysis of DDIs before clinical co-medications are administered. However, traditional DDI identification approaches (e.g., testing Cytochrome P450
\cite{veith2009comprehensive}
 or transporter-associated interactions
\cite{huang2007drug})
face challenges, such as high costs, long duration, animal welfare considerations
\cite {zhang2015label},
the very limited number of participants in the trial, and the great number of drug combinations under screening in clinical trials. As a result, only a few DDIs have been identified during drug development production (usually in the clinical trial phase). Some of them have been reported after drugs approved, and many have been found in post-marketing surveillance.

Computational approaches are a promising alternative to discovering potential DDIs on a large scale, and they have gained attention from academy and industry recently
\cite{wisniowska2016role, zhou2016simulation}.
Data mining-based computational approaches have been developed to detect DDIs from various sources
\cite{zhang2015label}
, such as scientific literature
\cite{bui2014novel, zhang2016leveraging}
, electronic medical records
\cite{yamanishi2008prediction}
, and the Adverse Event Reporting System of FDA (http://www.fda.gov). These approaches rely on post-market clinical evidence. So, they cannot provide alerts of potential DDIs before clinical medications are administered. In contrast, machine learning-based computational approaches (e.g. Naïve Similarity-Based Approach
\cite{vilar2014similarity}
, Network Recommendation-Based
\cite{zhang2015label}
, Classification-Based
\cite{cheng2014machine}
) can provide such alerts by utilizing pre-marketed or post-marketed drug attributes, such as drug features or similarities
\cite{pahikkala2015toward}.
These methods use different drug features to predict DDIs, such as chemical structures
\cite{vilar2014similarity}
, targets
\cite{luo2014ddi}
, hierarchical classification codes
\cite{cheng2014machine}
, side effects, and off-label side effects
\cite{zhang2015label, shi2017predicting}.

Most of these existing machine learning approaches are designed to predict the typical two-class problem, which only indicates how likely a pair of drugs is a DDI. However, two interacting drugs may change their own pharmacological behaviors or effects (e.g., increasing or decreasing serum concentration) in vivo. For example, the serum concentration of Flunisolide (DrugBank Id: DB00180) decreases when it is taken with Mitotane (DrugBank Id: DB00648), whereas its serum concentration increases when taken with Roxithromycin (DrugBank Id: DB00778). For short, the first case is degressive DDI, and the second case is enhancive DDI, which contains drug changes in terms of pharmacological effects. It is more important to know exactly whether the interaction increases or decreases the drug's pharmaceutical behaviors, especially when making optimal patient care, establishing drug dosage, designing prophylactic drug therapy, or finding the resistance to therapy with a drug
\cite{koch1981serum}.

On the other hand, the occurrence of both enhancive and degressive DDIs is not random, but most current approaches have not yet exploited this structural property and have been developed only for conventional two-classes DDIs. Furthermore, revealing such a structural relationship is very important because it can help us understand how DDIs occur. It is one of the most important steps for treating complex diseases and guides physicians in preparing safer prescriptions to high-order drug interaction. The proposed algorithms for predicting three-classes DDIs are introduced in the following. And how they work are briefly described. All three introduced algorithms use matrix factorization methods, which is a network recommender-based approach. The matrix factorization approach, with slightly modifying, is a suitable solution for the subject of predicting DDI that has received much attention from researchers.

In this paper, we firstly introduce data and features. Then, a novel algorithm (SNF-CNN) based on the integration of drug similarities and deep learning recommendation systems for predicting DDI is presented in a comprehensive three-class model. This algorithm is called Predicting Comprehensive Drug-Drug Interaction via Similarity Network Fusion and Convolutional Neural Networks.

The paper is organized as follows. In the first section, the data preparation process is explained. The recommendation system is then designed and trained on enhancive and degressive, which detects pairs of non-interacting drugs with high probability. Next, the previous recommender system, based on a convolutional neural network, is trained on incremental and decremental interaction data without interaction (detected in the previous step).In section Results and Discussions, we investigate the results of SNF-CNN in the 10-fold cross-validation process(10-fold CV).

It should be noted that the proposed method of this research is a recommender-based on deep neural networks and has no structural similarities with matrix factorization methods. The only reason for mentioning these methods is the limited number of articles that have used three-class data in their work.

\section*{Methods}
\subsection*{Dataset and features}
In this study, we use the data set presented in paper of \cite{yu2018predicting}. This set contains 568 approved small molecule drugs, each of them has at least one interaction with the other drugs in the set. In total, the interactions between these 568 drugs contain 21,351 DDIs, including 16,757 enhancive DDIs and 4,594 degressive DDIs.In addition, each drug represented as an  881-dimensional feature vector $ F_ {str} $  based on PubChem chemical structure descriptor and also a 9149-dimensional feature vector $ F_ {se} $ according to the off-label side effects provided by OFFSIDES.

\subsection*{Problem formulation}

Without loss of generality, let $D = \{d_i\}$, $i = 1, 2, .., m$ be a set of m approved drugs. Their  interactions can be accordingly represented as an $m \times m$ symmetric interaction matrix $A_{m \times m} = \{a_{ij}\}$. For the conventional DDIs, $a_{ij} = 1$ if $d_i$ interacts with $d_j$, and $a_{ij} = 0$ otherwise. For the comprehensive DDIs, $a_{ij} ∈ \{−1, 0, +1\}$ . Again, if $d_i$ and $d_j$ do not interact with each other, $a_{ij} = 0$ . When there is an enhancive DDI or a degressive DDI between $d_i$ and $d_j$ , $a_{ij} = +1$ or $a_{ij} = −1$ respectively.

In addition, each drug $d_i$ in the D is represented as a p-dimension feature vector $f_i = \left[  f_1, f_2, .., f_k, .., f_p \right]$, which $f_k = 1$ indicates the K-th specific chemical structure fragment or occurs an off-label side effect, and $f_k = 0$ otherwise. Because each drug has two chemical structure feature vectors and off-label side effects, there are two feature matrices of $F$ with dimensions of m×p (amount of p depends on kind of feature ). Matrices of $F_{str}$ and $F_{se}$ are, respectively, the feature matrix of the chemical structure and the feature matrix of off-label side effects.

\subsection*{Data preparing}

\subsubsection*{Cosine Similarity}








\subsubsection*{Sub-sub heading for section}
Text for this sub-sub-heading\ldots
\paragraph*{Sub-sub-sub heading for section}
Text for this sub-sub-sub-heading\ldots

In this section we examine the growth rate of the mean of $Z_0$, $Z_1$ and $Z_2$. In
addition, we examine a common modeling assumption and note the
importance of considering the tails of the extinction time $T_x$ in
studies of escape dynamics.
We will first consider the expected resistant population at $vT_x$ for
some $v>0$, (and temporarily assume $\alpha=0$)
%
\[
E \bigl[Z_1(vT_x) \bigr]=
\int_0^{v\wedge
1}Z_0(uT_x)
\exp (\lambda_1)\,du .
\]
%
If we assume that sensitive cells follow a deterministic decay
$Z_0(t)=xe^{\lambda_0 t}$ and approximate their extinction time as
$T_x\approx-\frac{1}{\lambda_0}\log x$, then we can heuristically
estimate the expected value as
%
\begin{equation}\label{eqexpmuts}
\begin{aligned}[b]
&      E\bigl[Z_1(vT_x)\bigr]\\
&\quad      = \frac{\mu}{r}\log x
\int_0^{v\wedge1}x^{1-u}x^{({\lambda_1}/{r})(v-u)}\,du .
\end{aligned}
\end{equation}
%
Thus we observe that this expected value is finite for all $v>0$ (also see \cite{koon,xjon,marg,schn,koha,issnic}).


\section*{Appendix}
Text for this section\ldots

%%%%%%%%%%%%%%%%%%%%%%%%%%%%%%%%%%%%%%%%%%%%%%
%%                                          %%
%% Backmatter begins here                   %%
%%                                          %%
%%%%%%%%%%%%%%%%%%%%%%%%%%%%%%%%%%%%%%%%%%%%%%

\begin{backmatter}

\section*{Acknowledgements}%% if any
Text for this section\ldots

\section*{Funding}%% if any
Text for this section\ldots

\section*{Abbreviations}%% if any
Text for this section\ldots

\section*{Availability of data and materials}%% if any
the code and data is available at GitHub page of \href{https://github.com/aminkhod/DDI-Project/tree/master/CNN\%20model\%20to\%20Recommend\%20Comperhancive\%20DDIs}{SNF-CNN code and data}


\section*{Ethics approval and consent to participate}%% if any
Text for this section\ldots

\section*{Competing interests}
The authors declare that they have no competing interests.

\section*{Consent for publication}%% if any
Text for this section\ldots

\section*{Authors' contributions}
Text for this section \ldots

\section*{Authors' information}%% if any
Text for this section\ldots

%%%%%%%%%%%%%%%%%%%%%%%%%%%%%%%%%%%%%%%%%%%%%%%%%%%%%%%%%%%%%
%%                  The Bibliography                       %%
%%                                                         %%
%%  Bmc_mathpys.bst  will be used to                       %%
%%  create a .BBL file for submission.                     %%
%%  After submission of the .TEX file,                     %%
%%  you will be prompted to submit your .BBL file.         %%
%%                                                         %%
%%                                                         %%
%%  Note that the displayed Bibliography will not          %%
%%  necessarily be rendered by Latex exactly as specified  %%
%%  in the online Instructions for Authors.                %%
%%                                                         %%
%%%%%%%%%%%%%%%%%%%%%%%%%%%%%%%%%%%%%%%%%%%%%%%%%%%%%%%%%%%%%

% if your bibliography is in bibtex format, use those commands:
\bibliography{bmc_article.bib}      % Bibliography file (usually '*.bib' )
\bibliographystyle{bmc-mathphys} % Style BST file (bmc-mathphys, vancouver, spbasic).

% for author-year bibliography (bmc-mathphys or spbasic)
% a) write to bib file (bmc-mathphys only)
% @settings{label, options="nameyear"}
% b) uncomment next line
%\nocite{label}

% or include bibliography directly:
% \begin{thebibliography}
% \bibitem{b1}
% \end{thebibliography}

%%%%%%%%%%%%%%%%%%%%%%%%%%%%%%%%%%%
%%                               %%
%% Figures                       %%
%%                               %%
%% NB: this is for captions and  %%
%% Titles. All graphics must be  %%
%% submitted separately and NOT  %%
%% included in the Tex document  %%
%%                               %%
%%%%%%%%%%%%%%%%%%%%%%%%%%%%%%%%%%%

%%
%% Do not use \listoffigures as most will included as separate files

\section*{Figures}
\begin{figure}[h!]
  \caption{Sample figure title}
\end{figure}

\begin{figure}[h!]
  \caption{Sample figure title}
\end{figure}

%%%%%%%%%%%%%%%%%%%%%%%%%%%%%%%%%%%
%%                               %%
%% Tables                        %%
%%                               %%
%%%%%%%%%%%%%%%%%%%%%%%%%%%%%%%%%%%

%% Use of \listoftables is discouraged.
%%
\section*{Tables}
\begin{table}[h!]
\caption{Sample table title. This is where the description of the table should go}
  \begin{tabular}{cccc}
    \hline
    & B1  &B2   & B3\\ \hline
    A1 & 0.1 & 0.2 & 0.3\\
    A2 & ... & ..  & .\\
    A3 & ..  & .   & .\\ \hline
  \end{tabular}
\end{table}

%%%%%%%%%%%%%%%%%%%%%%%%%%%%%%%%%%%
%%                               %%
%% Additional Files              %%
%%                               %%
%%%%%%%%%%%%%%%%%%%%%%%%%%%%%%%%%%%

\section*{Additional Files}
  \subsection*{Additional file 1 --- Sample additional file title}
    Additional file descriptions text (including details of how to
    view the file, if it is in a non-standard format or the file extension).  This might
    refer to a multi-page table or a figure.

  \subsection*{Additional file 2 --- Sample additional file title}
    Additional file descriptions text.

\end{backmatter}
\end{document}
